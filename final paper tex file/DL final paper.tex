%The current final version. (190726)
%SUBMITTED to ???, 2019.8.31.

%\documentclass[11pt, leqno]{article}
\documentclass[11pt]{article}
\usepackage{amsmath}
\usepackage{amsthm}
\usepackage{amssymb}
\usepackage{amsfonts}
\usepackage{longtable}
\usepackage{epsfig}
\usepackage{color}



\usepackage[utf8]{inputenc}
\usepackage[margin=1in]{geometry}
\usepackage{graphicx}
\usepackage{kotex}
\usepackage{mathtools}
\usepackage{setspace}
\usepackage{caption}
\usepackage{subcaption}
\usepackage{float}
\usepackage[table,xcdraw]{xcolor}
\usepackage{multirow}
\usepackage{afterpage}

\renewcommand{\baselinestretch}{1.2}
\newcommand{\mydoublespace}{\setlength{\baselineskip}{22pt}}
\hoffset=-25.5 truemm \voffset=-26. truemm
\setlength{\textheight}{8.5 true in} \setlength{\textwidth}{6.5true
in} \setlength{\oddsidemargin}{1.0 true in}
\setlength{\evensidemargin}{.25 true in}
\setlength{\topmargin}{0.615 true in} \setlength{\baselineskip}{1.75
\baselineskip}
%\def\theequation{\thesection.\arabic{equation}}
%\def\theequation{\thesection.\arabic{equation}}
\newtheorem{thm}{Theorem}[section]
\newtheorem{lm}{Lemma}[section]

%\mydoublespace
\newtheorem{lemma}{Lemma}
\newtheorem{theorem}{Theorem}
\newtheorem{proposition}{Proposition}
\newtheorem{assumption}{Assumption}
\newtheorem{remark}{Remark}
\newtheorem{corollary}{Corollary}
\newtheorem{claim}{Claim}

\renewcommand{\floatpagefraction}{0.1}



%============================================================================
% Title and Abstract
%============================================================================

\begin{document}

\begin{center}
{\bf\LARGE A Neural Network-based Clustering Methods for Statistical Arbitrage}
\vspace{.5cm}\\
 Dongwuk Kim, Seungkyu Kim, Chang Kyeom Kim \\
Department of Statistics, Seoul National University, Seoul, 08826,
Korea\\
\end{center}


\begin{abstract}
This paper aims to propose a novel hybrid of classical statistical arbitrage problem and deep learning framework. 
\end{abstract}

\noindent{\bf Key words}: Mean reversion, statistical arbitrage, deep learning, autoencoder, k-means clustering.


%============================================================================
% Section 1: Introduction
%============================================================================

\section{Introduction}

In this study, we aim to develop a method that ...



This paper is organized as follows. Section 2 reviews the basic mean-reversion strategy that were utilized developing the model. Section 3 proposes ...



%============================================================================
% Section 2: 
%============================================================================


\section{More new section}

Something to write down about.




%============================================================================
% Section 3: 
%============================================================================

\section{New section}
More thing to write down!


%============================================================================
% Section 4: 
%============================================================================

\section{Results}
Results!


%============================================================================
% Section 6: Concluding Remarks
%============================================================================


\section{Concluding remarks}
Conclusion!


%============================================================================
% References
%============================================================================


\newpage


\noindent {\bf \LARGE References}
\begin{description}

\item reference.



\end{description}





\end{document}
